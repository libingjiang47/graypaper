\section{Grandpa 与最佳链}\label{sec:bestchain}\label{sec:grandpa}

节点按照 \cite{stewart2020grandpa} 中定义的 \textsc{Grandpa} 协议参与。

\newcommand*{\final}{\natural}
\newcommand*{\best}{\flat}

我们将最新的最终确定区块定义为 $\block^\final$。所有与区块和状态相关的术语也都使用相同的上标。我们认为 \emph{最佳区块} $\block^\best$ 是从满足以下条件的可接受区块集合中选出的:

\begin{itemize}
  \item 拥有最终确定区块作为祖先。
  \item 不包含任何在相同时间槽(timeslot)上出现双重提议(两个有效区块)的未最终确定区块。
  \item 被认为已审计。
\end{itemize}

形式化定义如下:
\begin{align}
  \ancestors(\header^\best) &\owns \header^\final \\
  \isaudited^\best &\equiv \top \\
  \not\exists \header^A, \header^B &: \bigwedge \abracegroup[\,]{
    \header^A &\ne \header^B \\
    \header^A_\¬timeslot &= \header^B_\¬timeslot \\
    \header^A &\in \ancestors(\header^\best) \\
    \header^A &\not\in \ancestors(\header^\final)
  }
\end{align}

在这些可接受的区块中,应选择包含最多“其作者使用了密封密钥票据(seal-key ticket)而非后备密钥”的祖先区块的那个,作为最佳区块头,从而作为参与者在 \textsc{Grandpa} 协议中投票的链。

形式化表示,我们选择 $\block^\best$ 以最大化以下值 $m$:
\begin{equation}
  m = \sum_{\header^A \in \ancestors^\best} \isticketed^A
\end{equation}

在 \textsc{Grandpa} 中要投票的数据为最佳区块的区块头 $\block^\best$ 以及其 \emph{后验} 状态根 $\merklizestate{\thestate'}$。虽然状态根对 \textsc{Grandpa} 协议本身没有直接意义,但在投票/签名时与区块头一起包含,以确保使用 \textsc{Grandpa} 输出的系统能够验证尽可能最新的链状态。

这意味着在进行 \textsc{Grandpa} 投票以最终确定区块时,必须已知其后验状态。然而,由于 \textsc{Grandpa} 的主要作用是状态根验证,因此在没有对后验状态的承诺时最终确定一个区块是没有意义的。

后验状态仅在这样一个方面影响 \textsc{Grandpa} 投票:若对相同区块哈希的投票附带了不同的后验状态根,那么这些投票将被视为对不同区块的投票。这种情况仅会发生在节点作恶或本文档存在歧义的情况下。
