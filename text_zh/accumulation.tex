\newcommand*{\local¬fnsrmap}{P}
\newcommand*{\local¬fnservouts}{B}
\newcommand*{\local¬fngasused}{U}
\newcommand*{\local¬fnprovidable}{Y}
\newcommand*{\local¬fnprovide}{I}
\newcommand*{\local¬numberofreportsaccumulated}{n}
\newcommand*{\local¬servicegasused}{\mathbf{u}}

\section{累积}
\label{sec:accumulation}

累积可被定义为一个函数,其参数为 $\justbecameavailable$ 与 $\accounts$,以及部分(有时处于过渡中的)状态。该函数输出累积后的服务状态 $\accountspostpreimage$,并伴随更新后的状态元素 $\stagingset'$、$\authqueue'$ 与 $\privileges'$。

累积的基本命题其实很简单:我们只是希望执行那些至少拥有一个工作摘要(work-digest)的服务的 \emph{Accumulate} 逻辑,并向其传入相关摘要中的数据及必要的上下文信息。然而,这里存在三个主要复杂性。首先,我们必须定义此逻辑的执行环境,尤其是可用的宿主函数。其次,我们必须定义每个服务执行所允许的 gas 数量。最后,我们必须确定累积过程中转账的性质。


\subsection{历史与队列}

若某个工作报告存在尚未满足的依赖,则其累积将被延迟;若依赖无效,则其累积将被彻底取消。依赖以工作包哈希指定。为了确定哪些工作包已被累积,我们维护一个累积历史 $\accumulated$,其容量足以存放一个 epoch 内的工作报告。形式化为:
\begin{align}
  \label{eq:accumulatedspec}
  \accumulated &\in \sequence[\Cepochlen]{\protoset{\hash}} \\
  \accumulatedcup &\equiv \bigcup_{x \in \accumulated}(x)
\end{align}

我们还维护状态项 $\ready$,记录那些已就绪(即可用和/或已审计)但尚未累积的工作报告。这些报告在最多一个 epoch 前被提出,但其依赖尚未满足。除了工作报告本身,我们还保留其未累积的依赖(一个工作包哈希集合)。形式化为:
\begin{align}
  \label{eq:readyspec}
  \ready &\in \sequence[\Cepochlen]{\sequence{\tuple{\workreport, \protoset{\hash}}}}
\end{align}

新近可用的工作报告 $\justbecameavailable$ 被划分为两类:一类 $\justbecameavailable^!$ 没有任何前置依赖,可立即累积;另一类 $\justbecameavailable^Q$ 则需进入队列等待。形式化为:
\begin{align}
  \justbecameavailable^! &\equiv \sq{\build{r}{r \orderedin \justbecameavailable, \len{(r_\wr¬context)_\wc¬prerequisites} = 0 \wedge r_\wr¬srlookup = \emset}} \\
  \justbecameavailable^Q &\equiv E(\sq{
    D(r) \mid
    r \orderedin \justbecameavailable,
    \len{(r_\wr¬context)_\wc¬prerequisites} > 0 \vee r_\wr¬srlookup \ne \emset
  }, \accumulatedcup)\!\!\!\!\\
  D(r) &\equiv (r, \set{(r_\wr¬context)_\wc¬prerequisites} \cup \keys{r_\wr¬srlookup})
\end{align}

我们定义队列编辑函数 $E$,它是一个参数化于已累积工作包哈希集合的修改函数。其功能是:移除工作报告哈希已包含在参数集合中的条目,并移除依赖集中出现的哈希。形式化为:
\begin{equation}
  E\colon\abracegroup{
      &\tuple{\sequence{\tuple{\workreport, \protoset{\hash}}}, \protoset{\hash}} \to \sequence{\tuple{\workreport, \protoset{\hash}}} \\
    &\tup{\mathbf{r}, \mathbf{x}} \mapsto \sq{\build{
      \tup{r, \mathbf{d} \setminus \mathbf{x}}
    }{
      \begin{aligned}
        &\tup{r, \mathbf{d}} \orderedin \mathbf{r} ,\\
        &(r_\wr¬avspec)_\as¬packagehash \not\in \mathbf{x}
      \end{aligned}
    }}
  }
\end{equation}

我们进一步定义累积优先队列函数 $Q$,用于给出一组待累积的工作报告:
\begin{equation}
  Q\colon\abracegroup{
    &\sequence{\tuple{\workreport, \protoset{\hash}}} \to \workreports \\
    &\mathbf{r} \mapsto \begin{cases}
      \sq{} &\when \mathbf{g} = \sq{} \\
      \mathbf{g} \concat Q(E(\mathbf{r}, \local¬fnsrmap(\mathbf{g})))\!\!\!\! &\otherwise \\[3pt]
      \multicolumn{2}{l}{\,\where \mathbf{g} = \sq{\build{r}{\tup{r, \emset} \orderedin \mathbf{r}}}}
    \end{cases}
  }
\end{equation}


我们还定义映射函数 $\local¬fnsrmap$,用于从工作报告集合中提取其工作包哈希:
\begin{equation}
  \local¬fnsrmap\colon\abracegroup{
    \protoset{\workreport} &\to \protoset{\hash}\\
    \mathbf{r} &\mapsto \set{
      \build{(r_\wr¬avspec)_\as¬packagehash}{r \in \mathbf{r}}
    }
  }
\end{equation}

因此,本区块内可被累积的工作报告序列记作 $\justbecameavailable^*$:
\begin{align}
  \using m &= \H_\¬timeslot \bmod \Cepochlen\\
  \justbecameavailable^* &\equiv \justbecameavailable^! \concat Q(\mathbf{q}) \\
  \quad\where \mathbf{q} &= E(\concatall{\ready\interval{m}{}} \concat \concatall{\ready\interval{}{m}} \concat \justbecameavailable^Q, \local¬fnsrmap(\justbecameavailable^!))
\end{align}


\subsection{执行}
\label{sec:accumulationexecution}

每个区块内可用的 gas 总量有限,因此不一定能在单个区块内处理所有 $\justbecameavailable^*$ 项目。我们需要在两个互有矛盾的因素之间优化:  

1. 每个工作项有明确的 gas 限制,但实际使用的 gas 可能低于上限,只有执行后才能确定 → 这意味着执行模式趋向 **顺序化**。  
2. 由于 \textsc{pvm} 设置存在成本,我们希望尽可能将同一服务的多个工作项聚合到一次调用中 → 这意味着执行模式趋向 **并行化(服务聚合)**。  

解决方法是:定义函数 $\accseq$,顺序累积工作报告;并在其中调用 $\accpar$,以服务为单位进行聚合式并行累积。除第一次调用外,$\accseq$ 还会在后续调用中整合上轮累积产生的延迟转账(deferred-transfers)。因此,累积函数必须同时接受工作摘要与延迟转账信息。

为此,我们从工作摘要中提取关键信息,并结合工作报告隐含的信息,形成 \emph{操作元组} $\operandtuple$;同时定义延迟转账 $\defxfer$。形式化为:
\begin{align}
  \label{eq:operandtuple}
  \operandtuple &\equiv \tuple{
    \begin{alignedat}{5}
      \isa{\ot¬packagehash&}{\hash},\;
      \isa{&\ot¬segroot&}{\hash},\;
      \isa{&\ot¬authorizer&}{\hash},\;
      \isa{\ot¬payloadhash}{\hash},\;\\
      \isa{\ot¬gaslimit&}{\gas},\;
      \isa{&\ot¬authtrace&}{\blob},\;
      \isa{&\ot¬result&}{\blob \cup \workerror}
    \end{alignedat}
  }\\
  \label{eq:defxfer}
  \defxfer &\equiv \tuple{
    \isa{\dx¬source}{\serviceid} ,
    \isa{\dx¬dest}{\serviceid} ,
    \isa{\dx¬amount}{\balance} ,
    \isa{\dx¬memo}{\memo} ,
    \isa{\dx¬gas}{\gas}
  }\\
  \label{eq:accinput}
  \accinput &\equiv \operandtuple \cup \defxfer
\end{align}

其并集即为 \emph{累积输入} $\accinput$。

我们的形式化继续,定义 $\partialstate$ 来刻画(即能够表示)在累积过程中既需要又可变更的状态组件。这包括服务账户状态(如 $\accountspre$)、即将生效的验证者密钥 $\stagingset$、授权者队列 $\authqueue$,以及权限状态 $\privileges$。形式化为:
\begin{equation}
  \label{eq:partialstate}
  \partialstate \equiv \tuple{\begin{aligned}
    &\isa{\ps¬accounts}{\dictionary{\serviceid}{\serviceaccount}} \,,\;
    \isa{\ps¬stagingset}{\sequence[\Cvalcount]{\valkey}} \,,\;
    \isa{\ps¬authqueue}{\sequence[\Ccorecount]{\sequence[\Cauthqueuesize]{\hash}}} \,,\;
    \isa{\ps¬manager}{\serviceid} \,,\\
    &\isa{\ps¬assigners}{\sequence[\Ccorecount]{\serviceid}} \,,\;
    \isa{\ps¬delegator}{\serviceid} \,,\;
    \isa{\ps¬registrar}{\serviceid} \,,\;
    \isa{\ps¬alwaysaccers}{\dictionary{\serviceid}{\gas}}
  \end{aligned}}
\end{equation}

最后,我们定义 $\local¬fnservouts$ 与 $\local¬fngasused$,它们分别刻画按服务索引承诺的累积输出,以及按服务索引统计的 gas 使用量:
\begin{equation}
  \local¬fnservouts \equiv \protoset{\tuple{\serviceid, \hash}} \qquad
  \local¬fngasused \equiv \sequence{\tuple{\serviceid, \gas}}
\end{equation}

我们定义外层累积函数 $\accseq$。其将一个 gas 上限、延迟转账序列、工作报告序列、初始部分状态以及享有免费累积的服务字典,转换为:已累积的工作报告数量、累积后的部分状态、累积输出配对集合,以及按服务索引的 gas 使用量:
\begin{equation}
  \label{eq:accseq}
  \accseq\colon\abracegroup{
    &\tuple{\gas, \defxfers, \workreports, \partialstate, \dictionary{\serviceid}{\gas}} \to \tuple{\N, \partialstate, \local¬fnservouts, \local¬fngasused} \\
    &\tup{g, \mathbf{t}, \mathbf{r}, \psX, \mathbf{f}} \!\mapsto\! \begin{cases}
      \tup{0, \psX, \emset, \sq{}} &
        \when n = 0 \\
      \tup{i + j, \psX', \mathbf{b}^* \!\cup \mathbf{b}, \mathbf{u}^* \!\!\concat \mathbf{u}}\!\!\!\! &
        \text{o/w}\!\!\!\!\!\!\!\! \\
    \end{cases} \\
    &\quad\where i = \max(\Nmax{\len{\mathbf{r}} + 1}): \sum_{r \in \mathbf{r}\sub{\dots i}, d \in r_\wr¬digests}(d_\wd¬gaslimit) \le g \\
    &\quad\also n = \len{\mathbf{t}} + i + \len{\mathbf{f}} \\
    &\quad\also \tup{\psX^*\!\!, \mathbf{t}^*\!\!, \mathbf{b}^*\!\!, \mathbf{u}^*} = \accpar(\psX, \mathbf{t},\mathbf{r}\sub{\dots i}, \mathbf{f}) \\
    &\quad\also \tup{j, \psX'\!, \mathbf{b}, \mathbf{u}} = \accseq(g^* - \!\!\!\!\!\!\sum_{\tup{s, u} \in \mathbf{u}^*}\!\!\!\!\!\!(u), \mathbf{t}^*\!\!, \mathbf{r}\sub{i\dots}, \psX^*\!\!, \emset)\\
    &\quad\also g^* = g + \sum_{t \in \mathbf{t}}(t_\dx¬gas)
  }
\end{equation}

\newcommand*{\local¬accounts}{\mathbf{d}}
\newcommand*{\local¬stagingset}{\mathbf{i}}
\newcommand*{\local¬authqueue}{\mathbf{q}}
\newcommand*{\local¬acc}{\accumulate}
\newcommand*{\local¬manpost}{\mathbf{e}^*}
\newcommand*{\local¬fnreplaceifchanged}{R}

接下来定义并行化累积函数 $\accpar$。借助单服务累积函数 $\accone$,它将初始状态上下文、延迟转账序列、工作报告序列以及享有免费累积的服务字典,转换为:累积后的状态上下文、产生的新延迟转账与累积输出配对,以及按服务索引统计的 gas 使用量。注意:对于权限变更,我们使用函数 $R$ 来选择“管理者服务更改到的服务”,若无管理者更改,则选择“该服务自身更改到的服务”。这允许权限具备“所有权”,并有助于移除管理者服务(我们认为这是一种有益的可能)。形式化为:
\begin{equation}
  \label{eq:accpar}
  \accpar\colon\abracegroup[\;]{\begin{aligned}
    &\tuple{\partialstate, \defxfers, \workreports, \dictionary{\serviceid}{\gas}} \to \tuple{\partialstate, \defxfers, \local¬fnservouts, \local¬fngasused} \\
    &\tup{\psX, \mathbf{t}, \mathbf{r}, \mathbf{f}} \mapsto \tup{
      \tup{
        \ps¬accounts', \ps¬stagingset', \ps¬authqueue', \ps¬manager', \ps¬assigners', \ps¬delegator', \ps¬registrar', \ps¬alwaysaccers'
      }, \concatall{\mathbf{t}'}, \mathbf{b}, \mathbf{u}
    }\!\!\!\!\!\!\\
    &\text{where:}\\
    &\ \begin{aligned}
      \using \mathbf{s} &= \set{\build{
        d_\wd¬serviceindex
        }{
          r \in \mathbf{r}, d \in r_\wr¬digests
        }} \cup \keys{\mathbf{f}} \cup \set{\build{t_\dx¬dest}{t \in \mathbf{t}}} \\
      \local¬acc(s) &\equiv \accone(\psX, \mathbf{t}, \mathbf{r}, \mathbf{f}, s) \\
      \mathbf{u} &= \sq{\build{
          \tup{s, \local¬acc(s)_\ao¬gasused}
        }{
          s \orderedin \mathbf{s}
        }} \\
      \mathbf{b} &= \set{\build{
          \tup{s, b}
        }{
          s \in \mathbf{s},\,
          b = \local¬acc(s)_\ao¬yield,\,
          b \ne \none
        }} \\
      \mathbf{t}' &= \sq{\build{
          \local¬acc(s)_\ao¬defxfers
        }{
          s \orderedin \mathbf{s}
        }} \\
      \ps¬accounts' &= \local¬fnprovide(
        (\ps¬accounts \cup \mathbf{n}) \setminus \mathbf{m},
        \bigcup_{s \in \mathbf{s}} \local¬acc(s)_\ao¬provisions
      ) \\
      &\tup{
        \ps¬accounts, \ps¬stagingset, \ps¬authqueue, \ps¬manager, \ps¬assigners, \ps¬delegator, \ps¬registrar, \ps¬alwaysaccers
      } = \psX \\
      \local¬manpost &= \local¬acc(m)_\ao¬poststate \\
      \tup{\ps¬manager'\!,\ps¬alwaysaccers'} &=
        \local¬manpost_{\tup{\ps¬manager, \ps¬alwaysaccers}} \\
      \forall c \in \coreindex :
        \ps¬assigners'\sub{c} &= R(
          \ps¬assigners\sub{c},
          (\local¬manpost_\ps¬assigners)\sub{c},
          ((\local¬acc(\ps¬assigners\sub{c})_\ao¬poststate)_\ps¬assigners)\sub{c}
        ) \\
      \ps¬delegator' &= R(
        \ps¬delegator,
        \local¬manpost_\ps¬delegator,
        (\local¬acc(\ps¬delegator)_\ao¬poststate)_\ps¬delegator
      ) \\
      \ps¬registrar' &= R(
        \ps¬registrar,
        \local¬manpost_\ps¬registrar,
        (\local¬acc(\ps¬registrar)_\ao¬poststate)_\ps¬registrar
      ) \\
      \ps¬stagingset' &= (
          \local¬acc(\ps¬delegator)_\ao¬poststate
      )_\ps¬stagingset \\
      \forall c \in \coreindex :
        \ps¬authqueue'\sub{c} &= ((
          \local¬acc(\ps¬assigners\sub{c})_\ao¬poststate
        )_\ps¬authqueue)\sub{c} \\
      \mathbf{n} &= \bigcup_{s \in \mathbf{s}}(
        (\local¬acc(s)_\ao¬poststate)_\ps¬accounts
          \setminus
        \keys{\ps¬accounts \setminus \set{s}}
      ) \\
      \mathbf{m} &= \bigcup_{s \in \mathbf{s}}(
        \keys{\ps¬accounts}
          \setminus
        \keys{(\local¬acc(s)_\ao¬poststate)_\ps¬accounts}
      )
    \end{aligned}
  \end{aligned}}
\end{equation}
\begin{equation}
  R(o, a, b) \equiv \begin{cases}
    b &\when a = o \\
    a &\otherwise
  \end{cases}
\end{equation}
\undef{\local¬acc}
\undef{\local¬fnreplaceifchanged}

其中,$\local¬fnprovide$ 是“原像整合”函数:它将一个服务状态字典与一组(服务索引,blob)配对,转换为新的服务状态字典。对于已不存在的服务或已撤销相关请求的服务,所对应的原像提供将被忽略:
\begin{align}
  \local¬fnprovide &\colon\abracegroup{
    &\tuple{\dictionary{\serviceid}{\serviceaccount}, \protoset{\tuple{\serviceid, \blob}}} \to \dictionary{\serviceid}{\serviceaccount} \\
    &\tup{\mathbf{d}, \mathbf{p}} \mapsto \mathbf{d}'\;\where \mathbf{d}' = \mathbf{d}\;\text{except:} \\
    &\quad\forall \tup{s, \mathbf{i}} \in \mathbf{p},\;
      \local¬fnprovidable(\mathbf{d}, s, \mathbf{i}):\\
    &\qquad \mathbf{d}'\subb{s}_\sa¬requests\subb{\tup{\blake{\mathbf{i}}, \len{\mathbf{i}}}} =\sq{\thetime'}\\
    &\qquad \mathbf{d}'\subb{s}_\sa¬preimages\subb{\blake{\mathbf{i}}} = \mathbf{i}
  } \\
  \local¬fnprovidable &\colon\abracegroup{
    &\tuple{\dictionary{\serviceid}{\serviceaccount}, \serviceid, \blob} \to \bool \\
    &\tup{\mathbf{d}, s, \mathbf{i}} \mapsto \begin{cases}
      \mathbf{d}\subb{s}_\sa¬requests\subb{\tup{\blake{\mathbf{i}}, \len{\mathbf{i}}}} = \sq{} &\when s \in \keys{\mathbf{d}} \\
      \bot &\otherwise
    \end{cases}
  }
\end{align}

需要注意的是:当我们在上述语境中形成“所有已变更、新增与被移除的服务索引”的并集(记为 $\keys{\mathbf{n}} \cup \mathbf{m}$)时,不同服务不得对“同一个索引”同时提出“新增、变更或移除”。对于移除与变更集合,这种情况不会发生,因为可移除服务的代码哈希没有已知原像,因此它无法自行执行以进行更改。对于新增服务,这种情况也不应发生,因为新索引会被显式选择以避免冲突。若极少数情况下发生冲突,则该区块必须被判定为无效。

单服务累积函数 $\accone$ 将初始状态上下文、延迟转账序列、工作报告序列、享有免费累积的服务字典(其中值表示可免费使用的 gas 量)以及某个服务索引,转换为:变更后的状态上下文、\emph{转账} 序列、一个可能的累积输出、实际 \textsc{pvm} gas 使用量,以及一组原像提供。该函数从一组工作报告中梳理出属于某个服务的工作摘要,并据此调用 \textsc{pvm} 执行:
\begin{equation}
  \label{eq:acconeout}
  \acconeout \equiv \tuple{
    \begin{alignedat}{3}
      \isa{\ao¬poststate&}{\partialstate},\;
      \isa{&\ao¬defxfers&}{\defxfers},\;
      \isa{\ao¬yield}{\optional{\hash}},\;\\
      \isa{\ao¬gasused&}{\gas},\;
      \isa{&\ao¬provisions&}{\protoset{\tuple{\serviceid, \blob}}}
    \end{alignedat}
  }
\end{equation}
\begin{align}
  \label{eq:accone}
  &\accone \colon \abracegroup[\;]{
    &\begin{aligned}
      \tuple{\begin{aligned}
        &\partialstate, \defxfers, \workreports,\\
        &\dictionary{\serviceid}{\gas}, \serviceid
      \end{aligned}}
      &\to \acconeout \\
      \tup{\psX, \mathbf{t}, \mathbf{r}, \mathbf{f}, s} &\mapsto \Psi_A(\psX, \thetime', s, g, \mathbf{i}^T \!\!\concat \mathbf{i}^U)
    \end{aligned} \\
    &\text{where:} \\
    &\ \begin{aligned}
      g &= \subifnone{\mathbf{f}\sub{s}, 0}
        + \!\!\!\!\sum_{t \in \mathbf{t}, t_\dx¬dest = s}\!\!\!\!(t_\dx¬gas)
        + \!\!\!\!\!\!\!\!\sum_{r \in \mathbf{r}, d \in r_\wr¬digests, d_\wd¬serviceindex = s}\!\!\!\!\!\!\!\!(d_\wd¬gaslimit) \\
      \mathbf{i}^T &= \sq{\build{
        t
      }{
        t \orderedin \mathbf{t}, t_\dx¬dest = s
      }}\\
      \mathbf{i}^U &= \sq{\build{
        \tup{\begin{alignedat}{3}
          \is{\ot¬result}{d_\wd¬result},\,
          \is{\ot¬gaslimit}{d_\wd¬gaslimit},\,
          \is{\ot¬payloadhash}{d_\wd¬payloadhash},\,
          \is{&\ot¬authtrace\;&}{r_\wr¬authtrace&},\\
          \is{\ot¬segroot}{(r_\wr¬avspec)_\as¬segroot},\,
          \is{\ot¬packagehash}{(r_\wr¬avspec)_\as¬packagehash},\,
          \is{&\ot¬authorizer\;&}{r_\wr¬authorizer&}
        \end{alignedat}}
      }{
        \begin{alignedat}{2}
          r& \orderedin \mathbf{r},&\\
          d& \orderedin r_\wr¬digests,&\ d_\wd¬serviceindex = s
        \end{alignedat}
      }}
    \end{aligned}
  }\!\!\!\!
\end{align}

这里用到了 $\wd¬gaslimit$,即由所选延迟转账、工作报告与 gas 权益共同决定的 gas 上限。

\subsection{最终状态整合}

给定顶层 $\accseq$ 的结果,我们可以定义后验状态中的 $\privileges'$、$\authqueue'$ 与 $\stagingset'$,以及服务账户的第一中间状态 $\accountspostacc$ 与累积输出日志 $\lastaccout'$:
\begin{align}
  \nonumber
  &\using g = \max\left(
    \Cblockaccgas,
    \Creportaccgas \cdot \Ccorecount + \textstyle \sum_{x \in \values{\alwaysaccers}}(x)
  \right)\\
  \nonumber
  &\also \psX = \tup{
    \is{\ps¬accounts}{\accountspre},
    \is{\ps¬stagingset}{\stagingset},
    \is{\ps¬authqueue}{\authqueue},
    \is{\ps¬manager}{\manager},
    \is{\ps¬assigners}{\assigners},
    \is{\ps¬delegator}{\delegator},
    \is{\ps¬registrar}{\registrar},
    \is{\ps¬alwaysaccers}{\alwaysaccers}
  }
  \!\!\!\!\!\\
  \label{eq:finalstateaccumulation}
  &\tup{
    \local¬numberofreportsaccumulated, \psX', \mathbf{b}, \local¬servicegasused
  } \equiv \accseq(g, \sq{}, \justbecameavailable^*, \psX, \alwaysaccers) \\
  &\lastaccout' \equiv \sq{\tup{s, h} \in \mathbf{b}} \\
  \label{eq:accountspostaccdef}
  &\tup{
    \is\ps¬accounts{\accountspostacc},
    \is\ps¬stagingset{\stagingset'},
    \is\ps¬authqueue{\authqueue'},
    \is\ps¬manager{\manager'},
    \is\ps¬assigners{\assigners'},
    \is\ps¬delegator{\delegator'},
    \is\ps¬registrar{\registrar'},
    \is\ps¬alwaysaccers{\alwaysaccers'}
  } \equiv \psX'
  \!\!\!\!\!
\end{align}

由此,我们得到 $\local¬numberofreportsaccumulated$(本次累积的工作报告总数)与 $\local¬servicegasused$(各服务在累积过程中所用的 gas)。我们构造累积统计 $\accumulationstatistics$:它将被累积的服务索引映射为两项数据——累积全过程使用的 gas 以及被累积的工作项数量。形式化为:
\begin{align}
  \label{eq:accumulationstatisticsspec}
  &\accumulationstatistics \in \dictionary{\serviceid}{\tuple{\gas, \N}} \\
  \label{eq:accumulationstatisticsdef}
  &\textstyle \accumulationstatistics \equiv \set{\build{
    \kv{s}{\tup{G(s), N(s)}}
  }{
    G(s) + N(s) \ne 0
  }}
  \!\!\!\!\\
  \nonumber
  \where &G(s) \equiv \sum_{\tup{s, u} \in \mathbf{u}}(u) \\
  \nonumber
  \also &N(s) \equiv \len{\sq{\build{d}{
    r \orderedin \justbecameavailable^*\sub{\dots n} ,
    d \orderedin r_\wr¬digests ,
    d_\wd¬serviceindex = s
  }}} 
\end{align}

第二中间状态 $\accountspostxfer$ 定义如下:对所有已累积的服务更新其最近一次累积时间记录:
\begin{align}
  \accountspostxfer &\equiv \set{ \build{ \kv{s}{a'} }{ \kv{s}{a} \in \accountspostacc }} \\
  &\where a' = \begin{cases}
    a \exc a'_\sa¬lastacc = \thetime' &\when s \in \keys{\accumulationstatistics} \\
    a &\otherwise
  \end{cases}
\end{align}

我们通过整合本区块中被累积的工作报告,并从先前状态中“滚动”队列(最老的条目将被整体丢弃),得到最终的 ready 队列与 accumulated 映射:
\begin{align}
  \accumulated'_{\Cepochlen - 1} &= \local¬fnsrmap(\justbecameavailable^*\sub{\dots n}) \\
  \forall i \in \Nmax{\Cepochlen - 1}: \accumulated'\sub{i} &\equiv \accumulated\sub{i + 1} \\
  \forall i \in \N_\Cepochlen : \cyclic{\ready'}\sub{m - i} &\equiv \begin{cases}
    E(\justbecameavailable^Q, \accumulated'\sub{\Cepochlen - 1}) &\when i = 0 \\
    \sq{} &\when 1 \le i < \thetime' - \thetime \\
    E(\cyclic{\ready}\sub{m - i}, \accumulated'\sub{\Cepochlen - 1}) &\when i \ge \thetime' - \thetime
  \end{cases}
  \!\!\!\!
\end{align}

\subsection{原像整合}

在累积之后,我们必须整合查询外部交易(lookup extrinsic)中提供的所有原像,以得到最终的账户状态。查询外部交易是一个(服务索引,数据)配对的序列。该序列必须有序且无重复(见式 \ref{eq:preimagesareordered}),并且其中的数据必须是某个服务在“先前”状态中已请求但尚未提供的。形式化为:
\begin{align}
  \xtpreimages &\in \sequence{\tuple{ \serviceid,\, \blob }} \\
  \label{eq:preimagesareordered}\xtpreimages &= \sqorderuniqby{i}{i \in \xtpreimages} \\
  \forall \tup{\xp¬serviceindex, \xp¬data} &\in \xtpreimages : \local¬fnprovidable(\accountspre, \xp¬serviceindex, \xp¬data)
\end{align}

对于由于累积结果而不再有用的原像,我们将不作处理并直接忽略。我们将整合仍然相关的原像后得到的状态记为 $\accountspostpreimage$:
\begin{equation}
  \accountspostpreimage = \local¬fnprovide(\accountspostxfer, \xtpreimages)
\end{equation}

