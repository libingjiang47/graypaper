\begin{abstract}

% 无信任计算的协调扩展。

我们在此提出了 \Jam 的全面且形式化的定义,该协议结合了 \emph{Polkadot} 与 \emph{Ethereum} 的元素。在一个统一且一致的模型中,\Jam 提供了一个全局唯一、无需许可的对象环境——类似于以太坊率先提出的智能合约环境——并结合了通过可扩展节点网络实现的安全侧链并行计算,这一理念最早由波卡提出。

\Jam 引入了一种去中心化的混合系统,在一个安全且可扩展的“内核/链上”二元结构中提供智能合约功能。虽然其智能合约功能与以太坊范式存在一定相似之处,但整体服务模型主要由波卡的底层架构驱动。

\Jam 的本质是无需许可的,允许任何人将代码作为服务部署在其上,并支付与所消耗资源相称的费用。同时,可以通过采购与分配 \emph{core-time} 来触发该代码的执行。这里的 \emph{core-time} 是一种弹性且无处不在的计算度量,某种程度上类似于以太坊中的 Gas 购买机制。我们已经设想了一种与波卡兼容的 \emph{CoreChains} 服务。

\end{abstract}
