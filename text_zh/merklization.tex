\section{状态 Merkle 化}\label{sec:statemerklization}

Merkle 化过程定义了一种加密承诺,使得可以以简洁且快速的方式,证明状态中任意信息的真实性。我们分两个阶段描述该过程;第一步定义从 31 字节序列到(不受限)字节序列的映射,此过程称为 \emph{状态序列化}。第二步基于该映射生成一个 32 字节的承诺,此过程称为 \emph{Merkle 化}。

\subsection{序列化}

状态的序列化主要是将 $\thestate$ 的各个组成部分,放入一个从 31 字节序列 \emph{状态键(state-keys)} 到不定长字节序列的统一映射中。状态键由一个哈希分量与一个章节(chapter)分量构成,二者等价于某个状态组件的索引;若为 $\accounts$ 的内部字典,则为某个服务索引。

我们将状态键构造函数 $C$ 定义为:
\begin{equation}
  C\colon\abracegroup{
    \Nbits{8} \cup \tup{\Nbits{8}, \serviceid} \cup \tup{\serviceid, \blob} &\to \blob[31] \\
    i \in \Nbits{8} &\mapsto \sq{i, 0, 0, \dots} \\
    \tup{i, s \in \serviceid} &\mapsto \sq{i, n_0, 0, n_1, 0, n_2, 0, n_3, 0, 0, \dots}\ \where n = \encode[4]{s} \\
    \tup{s, h} &\mapsto \sq{n_0, a_0, n_1, a_1, n_2, a_2, n_3, a_3, a_4, a_5, \dots, a_{26}}\ \where n = \encode[4]{s}, a = \blake{h}
  }
\end{equation}

接着,状态序列化被定义为由各组件汇合而成的字典。由于密码学哈希的唯一性,若传入 $C$ 的输入不重复,则不会产生重复的状态键。形式化地,我们定义将某个状态 $\thestate$ 转换为其序列化形式的 $T$:
\begin{equation}
  T(\thestate) \equiv \abracegroup{
    &&C(1) &\mapsto \encode{\sq{\build{\var{x}}{x \orderedin \authpool}}} \;, \\
    &&C(2) &\mapsto \encode{\authqueue} \;, \\
    &&C(3) &\mapsto \encode{
      \var{\sq{\build{
        \tup{\rh¬headerhash, \rh¬accoutlogsuperpeak, \rh¬stateroot, \var{\rh¬reportedpackagehashes}}
      }{
        \tup{\rh¬headerhash, \rh¬accoutlogsuperpeak, \rh¬stateroot, \rh¬reportedpackagehashes} \orderedin \recenthistory
      }}},
      \mmrencode{\accoutbelt}
    } \;, \\
    &&C(4) &\mapsto \encode{
      \pendingset,
      \epochroot,
      \abracegroupboth{
        0\ &\when \sealtickets \in \sequence[\Cepochlen]{\safroleticket}\\
        1\ &\when \sealtickets \in \sequence[\Cepochlen]{\bskey}\\
      },
      \sealtickets,
      \var{\ticketaccumulator}
    } \;, \\
    &&C(5) &\mapsto \encode{
      \var{\sqorderby{x}{x \in \goodset}},
      \var{\sqorderby{x}{x \in \badset}},
      \var{\sqorderby{x}{x \in \wonkyset}},
      \var{\sqorderby{x}{x \in \offenders}}
    } \;, \\
    &&C(6) &\mapsto \encode{\entropy} \;, \\
    &&C(7) &\mapsto \encode{\stagingset} \;, \\
    &&C(8) &\mapsto \encode{\activeset} \;, \\
    &&C(9) &\mapsto \encode{\previousset} \;, \\
    &&C(10) &\mapsto \encode{
      \sq{\build{
        \maybe{\tup{\rs¬workreport, \encode[4]{\rs¬timestamp}}}
      }{
        \tup{\rs¬workreport, \rs¬timestamp} \orderedin \reports
      }}
    } \;, \\
    &&C(11) &\mapsto \encode[4]{\thetime} \;, \\
    &&C(12) &\mapsto \encode{
      \encode[4]{\manager, \assigners, \delegator, \registrar},
      \alwaysaccers
    } \;, \\
    &&C(13) &\mapsto \encode{
      \encode[4]{\valstatsaccumulator, \valstatsprevious},
      \corestats,
      \servicestats
    } \;, \\
    &&C(14) &\mapsto \encode{
      \sq{\build{
        \var{\sq{\build{
          \tup{\mathbf{r}, \var{\mathbf{d}}}
        }{
          \tup{\mathbf{r}, \mathbf{d}} \orderedin \mathbf{i}
        }}}
      }{
        \mathbf{i} \orderedin \ready
      }}
    } \;, \\
    &&C(15) &\mapsto \encode{
      \sq{\build{\var{\mathbf{i}}}{\mathbf{i} \orderedin \accumulated}}
    } \;, \\
    &&C(16) &\mapsto \encode{
      \var{\sq{\build{\tup{\encode[4]{s}, \encode{h}}}{\tup{s, h} \orderedin \lastaccout}}}
    } \;, \\
    \forall \kv{s}{\saX} \in \accounts: &&C(255, s) &\mapsto \encode{
      0,
      \saX_\sa¬codehash,
      \encode[8]{
        \saX_\sa¬balance,
        \saX_\sa¬minaccgas,
        \saX_\sa¬minmemogas,
        \saX_\sa¬octets,
        \saX_\sa¬gratis
      },
      \encode[4]{
        \saX_\sa¬items,
        \saX_\sa¬created,
        \saX_\sa¬lastacc,
        \saX_\sa¬parent
      }
    } \;, \\
    \forall \kv{s}{\saX} \in \accounts, \kv{\mathbf{k}}{\mathbf{v}} \in \saX_\sa¬storage:
      &&C(s, \encode[4]{2^{32}-1} \concat \mathbf{k}) &\mapsto \mathbf{v} \;, \\
    \forall \kv{s}{\saX} \in \accounts, \kv{h}{\mathbf{p}} \in \saX_\sa¬preimages:
      &&C(s, \encode[4]{2^{32}-2} \concat h) &\mapsto \mathbf{p} \;, \\
    \forall \kv{s}{\saX} \in \accounts, \kv{\tup{h, l}}{\mathbf{t}} \in \saX_\sa¬requests:
      &&C(s, \encode[4]{l} \concat h) &\mapsto \encode{
        \var{\sq{\build{\encode[4]{x}}{x \orderedin \mathbf{t}}}}
      }
  }
\end{equation}

注意:大多数行描述的是“由自然数导出的键”与“某状态组件的序列化值”之间的一对一映射。然而最后四行各自定义了一组映射,因为这些条目作用于所有服务账户;并且在最后三行中,键来自服务内部的嵌套字典。

还需注意:状态中所有“非判别(non-discriminator)”的数值类序列化,均依据该项的大小采用定长编码。

最后请注意:\Jam 不允许直接检查或枚举服务存储键。因此实现并不需要知道这些键的实际值;唯一重要的是用于 Merkle 化的“可序列化形式”,它是一个定长的哈希(与服务索引和项目标记一起)。实现可以据此避免存储这些键本身。

\subsection{Merkle 化}

在定义了 $T$ 之后,我们进一步给出 $\fnmerklizestate$ 的其余部分,其核心是将序列化映射转换为一个加密承诺。该承诺被定义为二叉 Patricia Merkle Trie 的根;其格式针对现代计算硬件做了优化,主要通过尺寸优化以更紧凑地适配典型内存布局,并减少不可预测的分支。

\subsubsection{节点编码与 Trie 标识}
我们使用根节点的哈希来标识(子)Trie,但有一个例外:空(子)Trie 使用零哈希 $\zerohash$ 标识。

节点固定为 512 位(64 字节)。每个节点不是分支(branch)就是叶子(leaf)。第一个比特用于区分这两种类型。

若为分支节点,剩余的 511 位在两个子节点哈希之间分配:0 位(左)子 Trie 的标识使用其末尾 255 位,1 位(右)子 Trie 的标识使用完整 256 位。

叶子节点进一步细分为“内嵌值叶(embedded-value leaf)”与“常规叶(regular leaf)”。节点的第二个比特用于区分二者。

若为内嵌值叶,则首字节余下 6 位用于存储内嵌值大小。接下来的 31 字节用于状态键,最后 32 字节用于值;若值长度不足 32 字节则以零填充。

若为常规叶,则首字节余下 6 位清零。随后 31 字节存储状态键,最后 32 字节存储该值的哈希。

形式化地,我们定义编码函数 $B$ 与 $L$:
\begin{align}
  B&\colon\abracegroup{
    \tuple{\hash, \hash} &\to \bitstring[512]\\
    \tup{l, r} &\mapsto \sq{0} \concat \text{bits}(l)\interval{1}{} \concat \text{bits}(r)
  }\\
  L&\colon\abracegroup{
    \tuple{\blob[31], \blob} &\to \bitstring[512]\\
    \tup{k, v} &\mapsto \begin{cases}
      \sq{1, 0} \concat \text{bits}(\encode[1]{\len{v}})\interval{2}{} \concat \text{bits}(k) \concat \text{bits}(v) \concat \sq{0, 0, \dots} &\when \len{v} \le 32\\
      \sq{1, 1, 0, 0, 0, 0, 0, 0} \concat \text{bits}(k) \concat \text{bits}(\blake{v}) &\otherwise
    \end{cases}
  }
\end{align}

据此,我们给出基本的 Merkle 化函数 $\fnmerklizestate$:
\begin{align}
  \merklizestate{\thestate} &\equiv M(\set{\build{\kv{\text{bits}(k)}{\tup{k, v}}}{\kv{k}{v} \in T(\thestate) }})\\
  M(d: \dictionary{\bitstring}{\tuple{\blob[31], \blob}}) &\equiv \begin{cases}
    \zerohash &\when \len{d} = 0\\
    \blake{\text{bits}^{-1}(L(k, v))} &\when \values{d} = \set{ \tup{k, v} }\\
    \blake{\text{bits}^{-1}(B(M(l), M(r)))} &\otherwise\\
    \multicolumn{2}{l}{\quad\where \forall b, p: \kv{b}{p} \in d \Leftrightarrow \kv{b\interval{1}{}}{p} \in \begin{cases}
      l &\when b_0 = 0 \\
      r &\when b_0 = 1
    \end{cases}
  }\end{cases}
\end{align}

\section{一般 Merkle 化}\label{sec:merklization}

\subsection{二叉 Merkle 树}

Merkle 树是一种密码学数据结构,可对一组有序值给出哈希承诺;其包含证明的计算复杂度为 $O(N)$,而包含证明的大小为 $O(\log(N))$。这种 \emph{良好平衡(well-balanced)} 的构造确保任意叶子的最大深度尽可能小,且该深度上的叶子数量也最小。

我们对 Merkle 树的底层函数定义为 \emph{节点} 函数 $N$,它接收长度为 $n$ 的若干 blob 序列,并返回一个 blob 或一个哈希:
\begin{equation}
  N\colon\abracegroup{
    \tuple{\sequence{\blob[n]}, \blob \to \hash} &\to \blob[n] \cup \hash \\
    \tup{\mathbf{v}, H} &\mapsto \begin{cases}
      \zerohash &\when \len{\mathbf{v}} = 0 \\
      \mathbf{v}_0 &\when \len{\mathbf{v}} = 1 \\
      H(\token{\$node} \concat N(\mathbf{v}_{\dots\ceil{\nicefrac{\len{\mathbf{v}}}{2}}}, H) \concat N(\mathbf{v}_{\ceil{\nicefrac{\len{\mathbf{v}}}{2}}\dots}, H)) &\otherwise
    \end{cases}
  }\label{eq:merklenode}
\end{equation}

细心的读者会注意到:若我们的 $\blob[n]$ 恰好等于 $\hash$,则该函数的输出将总是一个 $\hash$。但要保证安全性,必须避免原像碰撞的可能。为此我们加入哈希前缀 $\token{\$node}$ 以尽量降低此风险;只需确保序列中的各项在其他场景下使用不同前缀进行哈希,即可认为系统是安全的。

我们还定义 \emph{路径追踪} 函数 $T$,它会在从顶到叶的导航过程中,返回与目标路径相对的那些节点(即“相对侧”的兄弟节点)。该函数用于构造数据包含的证明。
\begin{equation}
  T\colon\abracegroup{
    \tuple{\sequence{\blob[n]}, \Nmax{\len{\mathbf{v}}}, \blob \to \hash}\ &\to \sequence{\blob[n] \cup \hash}\\
    \tup{\mathbf{v}, i, H} &\mapsto \begin{cases}
     \sq{N(P^\bot(\mathbf{v}, i), H)} \concat T(P^\top(\mathbf{v}, i), i - P_I(\mathbf{v}, i), H) &\when \len{\mathbf{v}} > 1\\
      \sq{} &\otherwise\\
      \multicolumn{2}{l}{
        \begin{aligned}
          \quad \where P^s(\mathbf{v}, i) &\equiv \begin{cases}
            \mathbf{v}_{\dots\ceil{\nicefrac{\len{\mathbf{v}}}{2}}} &\when (i < \ceil{\nicefrac{\len{\mathbf{v}}}{2}}) = s\\
            \mathbf{v}_{\ceil{\nicefrac{\len{\mathbf{v}}}{2}}\dots} &\otherwise
          \end{cases}\\[4pt]
          \quad \also P_I(\mathbf{v}, i) &\equiv \begin{cases}
            0 &\when i < \ceil{\nicefrac{\len{\mathbf{v}}}{2}} \\
            \ceil{\nicefrac{\len{\mathbf{v}}}{2}} &\otherwise
          \end{cases}\\
        \end{aligned}
      }
    \end{cases}\\
  }
\end{equation}

基于此,我们定义其他 Merkle 化函数。

\subsubsection{良好平衡树}
将良好平衡的二叉 Merkle 函数记为 $\fnmerklizewb$:
\begin{equation}
    \fnmerklizewb\colon \abracegroup{
      \label{eq:simplemerkleroot}
      \tuple{\sequence{\blob}, \blob \to \hash} &\to \hash \\
      \tup{\mathbf{v}, H} &\mapsto \begin{cases}
        H(\mathbf{v}_0) &\when \len{\mathbf{v}} = 1 \\
        N(\mathbf{v}, H) &\otherwise
      \end{cases} \\
    }
\end{equation}

该函数适用于对长度不大于 32 字节的数据创建证明的场景,因为它避免了对序列中每一项逐个哈希。若序列项更大,则应先对每一项进行哈希,以使证明尽可能小(因为每个证明通常会包含一个数据项本身)。

说明:当未显式提供哈希函数参数 $H$ 时,默认使用 Blake 2b。

\subsubsection{常深度树}
我们将常深度二叉 Merkle 函数记为 $\fnmerklizecd$。并定义两个与子树页面相关的函数:$\fnmerklejustsubpath{x}$ 与 $\fnmerklesubtreepage{x}$。后者返回单个叶子页(页面内各叶已进行“带前缀的数据”哈希),前者返回到该页面的 Merkle 路径。两者均假定页面大小为 $2^x$,并以页面索引为输入。
\begin{align}
  \label{eq:constantdepthmerkleroot}
  \fnmerklizecd&\colon \abracegroup{
    \tuple{\sequence{\blob}, \blob \to \hash} &\to \hash\\
    \tup{\mathbf{v}, H} &\mapsto N(C(\mathbf{v}, H), H)
  }\\
  \label{eq:constantdepthsubtreemerklejust}
  \fnmerklejustsubpath{x}&\colon \abracegroup{
    \tuple{\sequence{\blob}, \Nmax{\len{\mathbf{v}}}, \blob \to \hash} &\to \sequence{\hash}\\
    \tup{\mathbf{v}, i, H} &\mapsto T(C(\mathbf{v}, H), 2^xi, H)_{\dots\max(0, \ceil{\log_2(\max(1, \len{\mathbf{v}})) - x})}
  }\\
  \label{eq:constantdepthsubtreemerkleleafpage}
  \fnmerklesubtreepage{x}&\colon \abracegroup{
    \tuple{\sequence{\blob}, \Nmax{\len{\mathbf{v}}}, \blob \to \hash} &\to \sequence{\hash}\\
    \tup{\mathbf{v}, i, H} &\mapsto \sq{\build{H(\token{\$leaf} \concat l)}{l \orderedin \mathbf{v}_{2^xi \dots \min(2^xi+2^x, \len{\mathbf{v}})}}}
  }
\end{align}

为使上述 $\fnmerklejustsubpath{x}$ 的证明可接受,我们需假定验证方不仅知道给定索引项的值,还知道其所在 $2^x$ 大小子树中的所有其他叶子(由 $\fnmerklesubtreepage{x}$ 给出)。

同前,若未提供 $H$,可默认使用 Blake 2b。

在构造证明与计算 Merkle 根时,我们应用一个“常量化预处理器”函数 $C$:它以固定前缀“leaf”对所有数据项先行哈希,然后将整体长度向上填充至最近的 2 的整数次幂,不足部分以零哈希 $\zerohash$ 填充:
\begin{equation}
  C\colon\abracegroup{
    \tuple{\sequence{\blob}, \blob \to \hash} &\to \sequence{\hash}\\
    \tup{\mathbf{v}, H} &\mapsto \mathbf{v}' \ \where \abracegroup[\;]{
      \len{\mathbf{v}'} &= 2^{\ceil{\log_2(\max(1, \len{\mathbf{v}}))}}\\
      \mathbf{v}'\sub{i} &= \begin{cases}
        H(\token{\$leaf} \concat \mathbf{v}\sub{i}) &\when i < \len{\mathbf{v}}\\
        \zerohash &\otherwise \\
      \end{cases}
    }
  }
\end{equation}

\subsection{Merkle 山脉与腰带}\label{sec:mmr}

Merkle 山脉(Merkle Mountain Range,\textsc{mmr})是一种仅追加(append-only)的密码学数据结构,用于对一个值序列给出承诺。在该结构中,无论追加操作还是对某项的包含证明,时间与空间复杂度相对于集合规模均为 $O(\log(N))$。

我们将 Merkle 山脉定义为集合 $\sequence{\optional{\hash}}$,即一组“峰(peaks)”的序列。每个峰是一个包含 $2^i$ 个元素的 Merkle 树的根,其中 $i$ 为该峰在序列中的索引。由于我们支持非“$2^k-1$”大小的集合,某些峰可能为空,用 $\none$ 而非 Merkle 根表示。

由于直接使用哈希序列作为承诺较不便,Merkle 山脉通常会在发布前整体再哈希一次。对其哈希会阻止后续追加,因此需要在本地保留该山脉,以便生成未来的证明。

我们将 \textsc{mmb} 的追加函数记为 $\fnmmrappend$:
\begin{equation}
  \begin{aligned}
    \label{eq:mmrappend}
    \fnmmrappend&\colon\deffunc{
      \tuple{\sequence{\optional{\hash}}, \hash, \blob\to\hash} &\to \sequence{\optional{\hash}}\\
      \tup{\mathbf{r}, l, H} &\mapsto P(\mathbf{r}, l, 0, H)
    }\\
    \where P&\colon\deffunc{
      \tuple{\sequence{\optional{\hash}}, \hash, \N, \blob\to\hash} &\to \sequence{\optional{\hash}}\\
      \tup{\mathbf{r}, l, n, H} &\mapsto \begin{cases}
        \mathbf{r} \append l &\when n \ge \len{\mathbf{r}}\\
        R(\mathbf{r}, n, l) &\when n < \len{\mathbf{r}} \wedge \mathbf{r}\sub{n} = \none\\
        P(R(\mathbf{r}, n, \none), H(\mathbf{r}\sub{n} \concat l), n + 1, H) &\otherwise
      \end{cases}
    }\\
    \also R&\colon\deffunc{
      \tuple{\sequence{T}, \N, T} &\to \sequence{T}\\
      \tup{\mathbf{s}, i, v} &\mapsto \mathbf{s}'\ \where \mathbf{s}' = \mathbf{s} \exc \mathbf{s}'\sub{i} = v
    }
  \end{aligned}
\end{equation}

我们将 \textsc{mmr} 的编码函数记为 $\fnmmrencode$:
\begin{equation}
  \fnmmrencode\colon\deffunc{
    \sequence{\optional{\hash}} &\to \blob \\
    \mathbf{b} &\mapsto \encode{\var{\sq{\build{\maybe{x}}{x \orderedin \mathbf{b}}}}}
  }
\end{equation}

我们将 \textsc{mmr} 的“超级峰(super-peak)”函数记为 $\fnmmrsuperpeak$:
\begin{equation}
  \fnmmrsuperpeak\colon\deffunc{
    \sequence{\optional{\hash}} &\to \hash \\
    \mathbf{b} &\mapsto \begin{cases}
      \zerohash &\when \len{\mathbf{h}} = 0\\
      \mathbf{h}_0 &\when \len{\mathbf{h}} = 1\\
      \keccak{\token{\$peak} \concat \mmrsuperpeak{\mathbf{h}_{\dots\len{\mathbf{h}}-1}} \concat \mathbf{h}_{\len{\mathbf{h}}-1}} &\otherwise \\
      \multicolumn{2}{l}{\where \mathbf{h} = \sq{\build{h}{h \orderedin \mathbf{b}, h \ne \none}}}
    \end{cases}
  }
\end{equation}
