\section{结论}
\label{sec:conclusion}

我们提出了一种全新的计算模型,它能够利用现有的加密经济机制,在不引发持久性状态碎片化(state-fragmentation)并牺牲整体一致性的情况下,实现大幅度的可扩展性提升。我们将这一整体模式称为“收集—精炼—连接—累积(collect-refine-join-accumulate)”。此外,我们形式化定义了该逻辑的链上部分,即“连接—累积(join-accumulate)”部分。我们将该协议称为 \Jam 链。

我们认为,\Jam 的模型提供了一种全新的“最佳平衡点”:与完全同步模型相比,它能够在安全、弹性的共识下完成大量计算;同时,与长期碎片化模型不同,它仍能对计算的时序性与整合进入单一状态机提供严格保证。

\subsection{进一步工作}

虽然我们能够基于一些基本假设来估算理论计算能力,甚至与现有系统进行粗略对比,但实际数据仍然极其宝贵。我们认为该模型值得进一步的实证研究,以便更好地理解这些理论极限如何转化为现实中的性能表现。我们也认为,进行恰当的成本分析与与现有协议的比较,将是一个极具价值的进一步研究主题。

我们有充分理由相信,\Jam 的设计能够托管一种服务,在该服务下可以验证 Polkadot 的 \emph{平行链(parachains)},然而仍需要更多原型工作来理解基于 \textsc{pvm} 的计量系统能够支持的潜在吞吐量。我们将这一报告留作后续工作。同样地,我们也有意省略了高层协议元素的细节,包括加密货币、核心时间(coretime)销售、质押以及常规的智能合约功能。

为使协议的实际应用更加便捷,我们正在考虑一些潜在的修改,包括:

\begin{itemize}
  \item 在累积阶段允许服务之间进行同步调用。
  \item 对 \texttt{transfer} 函数施加限制,以便在累积阶段实现更大规模的并行化。
  \item 在特定条件下,允许在累积阶段保留大量额外的计算能力。
  \item 在工作包(Work Package)格式中引入默克尔化(Merklization),以避免必须下载整个包来验证其授权。
\end{itemize}

网络协议在此阶段也有意未作定义,其描述必须在后续提案中完成。

目前,验证者性能尚未在链上追踪。我们预计在 \Jam 协议的最终版本中将会对其进行链上追踪,但其具体格式尚未确定,因此在此暂时省略。
