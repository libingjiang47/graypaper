\section{担保}\label{sec:guaranteeing}

对工作包的担保涉及创建并分发相应的 \emph{工作报告(work-report)},并要求满足特定条件。除了报告之外,还需要附带一个签名,用以证明验证者对其正确性的承诺。在拥有两个担保人签名后,该工作报告即可分发给即将出块的 \Jam 链区块作者,以便用于 $\xtguarantees$,从而使担保人获得奖励。

我们假设在一个公开系统中,如果验证者出现故障并承诺了一份不能如实反映 $\computereport$ 在工作包上运行结果的报告,将会受到严厉惩罚。总体过程如下:

\begin{enumerate}
    \item 评估工作包的授权,并与最新 \Jam 链状态中的授权池进行交叉验证。
    \item 创建并发布工作包报告。
    \item 按照纠删码方案对工作包及其外部数据与导出数据进行分块。
    \item 将上述分块分发至验证者集合。
    \item 在有请求时向其他验证者提供工作包、外部数据及导出数据,这对优化网络性能也很有帮助。
\end{enumerate}

对于我们收到的任意工作包 $p$,我们可以确定其对应的工作报告(若存在),该报告归属于我们被分配的核心 $c$。在需要 \Jam 链状态时,我们始终使用最近区块的链状态。

对于分配到核心 $c$ 的索引为 $v$ 的担保人和工作包 $p$,其工作报告 $r$ 定义为:
\begin{equation}
  r = \computereport(p, c)
\end{equation}

此类担保人可以安全地创建并分发载荷 $\tup{s, v}$。其中的组件 $s$ 可依据公式 \ref{eq:guarantorsig} 生成;具体而言,它是使用验证者注册的 Ed25519 密钥对载荷 $l$ 的签名:
\begin{equation}
  l = \blake{\encode{r}}
\end{equation}

为了最大化收益,担保人应确保工作摘要满足在 \ref{sec:workreportguarantees} 节担保外部调用中规定的全部要求。这包括上下文有效性以及授权池中的授权包含。若不满足这些条件虽然不会导致惩罚,但会使区块作者无法将该工作包纳入区块,从而减少奖励。

高级节点可以尝试预测在报告抵达区块作者时的链状态,以最大化其报告被上链的可能性。朴素节点则可以直接使用当前链头来验证工作报告。为了减少计算工作量,节点应当在调用 $\Psi_R$ 函数计算报告工作结果 \emph{之前},完成所有此类验证。

一旦评估某个工作包适合担保,担保人应当尝试在核心内部达成共识,以最大化其工作不被浪费的概率。为此,他们应将该工作包发送给同一核心上尚未知晓该工作包的其他担保人。

为了减少区块作者的工作量、从而最大化预期收益,担保人应尝试基于工作报告、核心索引和尽可能多的证明(包括自己的)来构造其核心的下一次担保外部调用。

为了减少区块作者因反垃圾措施而忽略担保人的概率,担保人应在每个时隙签署平均不超过两个工作报告。
