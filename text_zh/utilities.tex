\section{洗牌}\label{sec:shuffle}

Fisher–Yates 洗牌函数形式化定义如下:
\begin{equation}\label{eq:suffle}
  \forall T, l \in \N: \fnfyshuffle\colon\abracegroup{
    \tuple{\sequence[l]{T}, \sequence[l:]{\N}} &\to \sequence[l]{T}\\
    \tup{\mathbf{s}, \mathbf{r}} &\mapsto \begin{cases}
      \sq{\mathbf{s}_{\mathbf{r}_0 \bmod l}} \concat \fyshuffle{\mathbf{s}'\sub{\dots l-1}, \mathbf{r}\sub{1\dots}}\ \where \mathbf{s}' = \mathbf{s} \exc \mathbf{s'}\sub{\mathbf{r}_0 \bmod l} = \mathbf{s}\sub{l - 1} &\when \mathbf{s} \ne \sq{}\\
      \sq{} &\otherwise
    \end{cases}
  }
\end{equation}

由于经常需要基于某个以哈希表示的随机种子来对序列进行洗牌,我们提供了洗牌函数 $\fnfyshuffle$ 的另一个形式:它接受一个 32 字节的哈希,而不是数值序列。我们定义“由哈希生成数列”的函数 $\fnseqfromhash{}$ 如下:
\begin{align}
  \forall l \in \N:\ \fnseqfromhash{l}&\colon\abracegroup{
    \hash &\to \sequence[l]{\Nbits{32}}\\
    h &\mapsto \sq{\build{
      \decode[4]{\blake{h \concat \encode[4]{\floor{\nicefrac{i}{8}}}}
      \sub{4i \bmod 32 \dots+4}}
    }{
      i \orderedin \N\sub{l}
    }}
  }\\
  \label{eq:sequencefromhash}
  \forall T, l \in \N:\ \fnfyshuffle&\colon\abracegroup{
    \tuple{\sequence[l]{T}, \hash} &\to \sequence[l]{T}\\
    \tup{\mathbf{s}, h} &\mapsto \fyshuffle{\mathbf{s}, \seqfromhash{l}{h}}
  }
\end{align}
