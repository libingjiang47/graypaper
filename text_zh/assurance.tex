\section{可用性担保}\label{sec:assurance}

当验证者持有某一工作报告(work-report)对应的全部纠删码分片时,应当发布一份签名声明,称为 \emph{担保(assurance)},用于表明该报告当前处于待可用状态。为了使某个工作报告获得担保,验证者必须掌握两类数据:

首先,是该报告捆绑包(bundle)的纠删码分片。该分片的有效性可以通过工作报告的工作包纠删码根(work-package erasure-root)以及其在正确位置上的 Merkle 包含证明来轻易验证。该证明应由担保方提供。此分片用于验证工作报告的有效性与完整性,在报告被认为已审计之后即可不再保留。在此之前,它应当在验证者请求时予以提供。

其次,验证者还应当持有由 \emph{segments root} 引用的每个导出片段(exported segment)对应的纠删码分片。这些分片应当至少保留 28 天,并在任何验证者请求时提供。
