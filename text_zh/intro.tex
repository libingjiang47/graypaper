\section{引言}

\subsection{命名法}

%正如其同名论文所指出的,\emph{Polkadot} 这个名字在不同语境中具有多重含义。它可能指代该论文中提出的可扩展异构多链的产品愿景;也可能指代随之出现的去中心化区块链网络及其状态与历史;或者指代用于维护并运行该网络及其他如 \emph{Kusama} 的协议与技术基础。

在本文中,我们介绍一个去中心化、加密经济协议,Polkadot 网络将基于治理机制的批准,在一次重大升级中过渡到该协议。

该协议的早期、未成熟版本最早在 Polkadot Fellowship \textsc{rfc}\oldstylenums{31} 中提出,称为 \emph{CoreJam}。CoreJam 之名源自其核心服务命题中的 collect/refine/join/accumulate 计算模型。尽管 CoreJam \textsc{rfc} 所建议的仅是对 Polkadot 协议的一次不完整、范围有限的修改,但 \Jam 指代的是一个完整且一致的整体区块链协议。

\subsection{驱动因素}

在区块链及更广泛的 Web3 领域,我们的首要驱动力是提供弹性(resilience)。一个合格的 Web3 数字系统应当遵守其声明的服务特征——并且理想情况下,甚至满足用户的预期——无论任何经济参与者(包括个人、组织乃至其他 Web3 系统)的意愿、财富或权力如何。当然,这一目标是理想化的,我们必须在实践中保持务实,承认其交付程度的局限。然而,一个 Web3 系统应当力求提供如此强有力的保证,以至于在实际意义上可被描述为 \emph{不可阻挡}。

或许,Bitcoin 是此类系统在经济领域中的首个实例,但其所提供的服务性质并非通用。基于规则的服务,其效用仅取决于可以被构想并置入其中的规则的通用性。Bitcoin 的规则允许最初的应用场景,即固定发行的代币,其所有权通过秘密的掌握来近似保证并自动执行,同时也支持一些进一步的扩展。

随后,Ethereum 提供了一个类别上更加通用的规则集,实际上接近图灵完备。\footnote{Gas 机制确实限制了可在其上执行的程序,通过为可执行步数设置上限来避免无限计算,但在无许可环境中必须引入某种限制。} 在我们旨在构建一个大规模多用户应用平台的 Web3 背景下,通用性至关重要,因此我们将其视为基础假设。

除了弹性与通用性之外,还存在更深层次的因素。为当前目标,我们识别出三项额外目标:
\begin{enumerate}
  \item \label{enum:resilience} 弹性(Resilience):高度抗拒停止、破坏与审查。
  \item \label{enum:generality} 通用性(Generality):能够执行图灵完备计算。
  \item \label{enum:performance} 性能(Performance):能够快速且低成本地执行计算。
  \item \label{enum:coherency} 一致性(Coherency):状态中不同元素间存在因果关系,从而支持应用的良好组合性。
  \item \label{enum:accessibility} 可及性(Accessibility):创新的壁垒极低;简单、快速、廉价且无许可。
\end{enumerate}

作为已声明的 Web3 技术,我们对前两项目标作出隐含假设。有趣的是,\ref{enum:performance} 与 \ref{enum:coherency} 两项目标在信息论原则下存在对立关系——我们相信这一原则已有对应的表述,但尚未知其名称。为便于讨论,我们将其称为 \emph{规模-一致性对立}(size-coherency antagonism)。

\subsection{规模-一致性对立下的扩展}

规模-一致性对立的原则很简单:随着信息系统状态空间的扩张,系统必然变得不再一致。这直接源于“因果关系受限于速度”的原理。物理学允许的最高速度是 $C$(真空中的光速),但其他信息系统会受到更低的约束:在生物系统中主要由化学过程决定,在电子系统中则由电子在不同物质中的速度决定。分布式软件系统的约束通常更低,因为其依赖于由软件、硬件与分组交换网络组成的底层结构,且这些网络的可靠性各异。

其推理过程为:
\begin{enumerate}
  \item 系统用于数据处理的状态越多,占据的空间就越大。
  \item 占据的空间越大,状态组件之间的平均距离与方差就越大。
  \item 随着平均值与方差的增加,因果解析的时间(即事件的所有正确推论得以传递的时间)在系统中变得分散,从而导致不一致性。
\end{enumerate}

暂且不考虑整体安全性问题,我们可以通过将系统划分为因果独立的子系统来管理不一致性,每个子系统足够小以保持一致。在资源充足的环境下,细菌会分裂成两个个体,而非增长至双倍大小。这种模式是一种粗糙的方式来应对扩展下的不一致性:系统内部处理保持低规模和完全一致,系统间处理则支持更大的总体规模,但缺乏一致性。这是 Polkadot、Cosmos 以及以太坊主流扩展愿景背后的原理(稍后将详细讨论)。这些系统通常依赖异步且简化的通信,并借助“结算区域”提供一个小范围的一致状态空间,用于处理诸如代币转移等特定交互。

本研究探讨了对立关系中的中间地带,避免像现有方法那样持久性地碎片化系统的状态空间。我们引入一种新的计算模型,将高度可扩展的 \emph{大部分一致} 元素通过流水线与同步的、完全一致的元素结合。异步性并未被消除,而是被限制在流水线长度内,并以类似多核 \textsc{cpu} 与共享 \textsc{ram} 系统中常见的“缓存亲和性”替代了现有可扩展系统中的粗糙分区。

不同于基于 \textsc{snark} 的二层扩展技术,这一模型依赖于加密经济机制,继承了其低成本和高性能的特征,同时避免了对中心化的偏向。

\subsection{文档结构}

我们将在第 \ref{sec:previouswork} 节简要概述区块链技术中现有的扩展方法。在第 \ref{sec:notation} 节中,我们将定义并澄清本文形式化描述所用的符号。

接着,在第 \ref{sec:overview} 节中,我们对协议进行总体概述,涵盖主要部分,包括 Polkadot 虚拟机(\textsc{pvm})、共识协议 Safrole 与 \textsc{Grandpa}、公共时钟,并为形式化奠定基础。

随后,我们给出完整的协议定义,分为两部分:其一是正确的链上状态转移公式,供所有希望验证链状态的节点使用;其二是在第 \ref{sec:workpackagesandworkreports} 节和第 \ref{sec:bestchain} 节中描述持有验证者密钥的行为者在链下的诚实策略。

正文部分最后在第 \ref{sec:discussion} 节中讨论该协议的性能特征,并在第 \ref{sec:conclusion} 节中作结。

附录包含对协议定义同样重要的补充材料,包括 \textsc{pvm}(见附录 \ref{sec:virtualmachine} 与 \ref{sec:virtualmachineinvocations})、序列化与 Merkle 化(见附录 \ref{sec:serialization} 与 \ref{sec:statemerklization})、以及密码学(见附录 \ref{sec:merklization}、\ref{sec:bandersnatch} 与 \ref{sec:erasurecoding})。最后在附录 \ref{sec:definitions} 中给出术语索引,涵盖文中使用的所有简单常数项,文末附参考文献。
